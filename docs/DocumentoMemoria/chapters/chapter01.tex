% cap1.tex

\chapter{Introducción}
\label{c1} % la etiqueta para referencias

En la actualidad, se reconoce a nivel mundial que el planeta está experimentando un cambio climático potencialmente perjudicial para la vida en él. También, múltiples estudios científicos y gran parte de la sociedad concuerdan en que muchas de las actividades del ser humano son un acelerador significativo de este cambio y el aumento de la temperatura. De acuerdo con diversos estudios, esto último apunta a que el aumento global de temperatura se debe al efecto invernadero, principalmente producido por $CO_2$. Este es un componente generado en un gran volumen por las actividades humanas tales como el transporte, la generación de energía, la ganadería, entre otros.
\vspace{2.5mm}

Esta situación ha generado un incentivo en las personas para encontrar soluciones y reducir el factor humano en el cambio climático. Con esto, se han producido importantes innovaciones y cambios en las industrias mencionadas y también se ha impulsado la creación de sistemas que permitan que estas transiciones sean sustentables y realistas en el largo plazo. 
\vspace{2.5mm}

Específicamente en Chile, la Estrategia climática de largo plazo de Chile (ECLP) presentada en el año 2021 por el Ministerio del Medio Ambiente concluye que el sector energético representó un 77\% de los gases de efecto invernadero producidos  en el año 2018 y, en ese sector, la generación de energía representa el mayor aporte con un 30\% del total de emisiones. Con el fin de disminuir este porcentaje, existen alternativas de generación más limpia que en la actualidad se emplean mayoritariamente. Sin embargo su implementación y producción es aún, en la mayoría de los casos, menos rentable, aunque la trayectoria de estas tecnologías muestra que, potencialmente, esto cambiará de forma positiva.
\vspace{2.5mm}

No obstante el trabajo realizado por \citeB{amigo_two_2021} demuestra que, con el sistema actual de penalización de emisiones que regula la industria generadora de electricidad, no es suficiente para lograr un cambio significativo y la reducción necesaria para lograr las metas prepuestas por Chile en el Acuerdo de París ni en el ECLP. Por lo tanto, los autores proponen un sistema de \emph{cap and trade} de dos etapas con mercado de compra y venta de permisos de contaminación. En este, las metas sí se logran cumplir y demuestra cómo es el efecto de este sistema con distintos presupuestos de contaminación. Sin embargo, este modelo propuesto presenta algunas limitaciones y posibilidades de mejora, que este trabajo busca aportar, específicamente sobre el problema del \emph{auctioneer} (subastador en inglés) de estos permisos de contaminación.


\section{Motivación}
Encontrar un modelo óptimo y eficiente para implementar en la industria de energía parece ser una tarea difícil. Muchos países tienen la intención de rebajar su aporte de contaminante global para alcanzar el acuerdo de París, pero una iniciativa no es suficiente, si no se respalda con continuo cambio y mejora. Cualquier tipo de modelo implementado a gran escala, como la fiscalización de la industria eléctrica, requiere de evaluación continua y estudio de implementaciones en otros países para encontrar cuál es la mejor solución al problema.
\vspace{2.5mm}

El ECLP dicta que Chile tiene como objetivo ser carbono neutral\footnote{Carbono neutral no significan 0 toneladas de dióxido de carbono emitidas, sino que las emisiones sean iguales a las absorciones por agentes como el sector forestal.} como fecha límite para el año 2050. A la fecha, el país tiene como estrategia para disminuir las emisiones de la industria eléctrica aplicar un ``impuesto verde'' descrito en la ley 20.780 promulgada en el año 2014. Diversos estudios (mencionados en el marco teórico de este trabajo) argumentan que esto no es suficiente y su aplicabilidad no es sustentable y no incentiva el cambio a energías renovables. En efecto, los autores \citeB{amigo_two_2021} demuestran y concluyen que, en primer lugar, el objetivo de carbono neutralidad es lograble y, en segundo lugar, que el \emph{National Determined Contribution}  (NDC por sus siglas en inglés)\footnote{National Determined Contribution: Aporte del país en la propuesta del acuerdo de París}  propuesto para el acuerdo de París\footnote{En el acuerdo de París se estableció como objetivo no superar un aumento global de temperatura de 2 °C para finales de este siglo.}, que consiste en que, para el año 2030, el nivel de $CO_2 e$ (dióxido de carbono emitido) no supere los 131 $MtCO_2 e$ en Chile, es una meta fácil de lograr y, por lo tanto, ineficiente. Ellos proponen la implementación de una versión de su modelo de \emph{cap and trade}, con la cuál se pueden alcanzar objetivos mucho más ambiciosos en la reducción de emisiones y sobre el cambio a una generación de energía con fuentes sustentables.
\vspace{2.5mm}

Entonces, el propósito de este trabajo es aportar en la investigación, testeo y complementación del modelo por medio de la incorporación del concepto de atención racional.
\vspace{2.5mm}

La finalidad es entender su efecto en el modelo y concluir si este presenta potencialidad para disminuir las emisiones de carbono y así aumentar el bienestar social.


\section{Objetivos}
\subsection{Objetivo general}
El objetivo general de este trabajo es reformular el modelo de \textit{cap and trade} de dos etapas creado por \citeB{amigo_two_2021}  por medio de la incorporación del concepto de atención racional  y evaluar su efecto en las emisiones de carbono en la industria eléctrica . 

\subsection{Objetivos específicos}
Los objetivos específicos necesarios para lograr el objetivo general son:

\begin{enumerate}
\item Identificar modelos de equilibrio y modelos de equilibrio en capacidad.
\item Identificar métodos de resolución de MCP.
\item Programar problemas MCP en GAMS.
\item Replicar el modelo original de \citeB{amigo_two_2021}.
\item Implementar atención racional en el modelo.
\item Resolver el nuevo modelo y analizar resultados.
\end{enumerate}


\section{Alcances}

Los alcances de este trabajo son: estudiar con profundidad los modelos de equilibrio en capacidad. Resolver problemas complejos de tipo MCP como mínimo en un lenguaje de programación y un \textit{solver}. Replicar el modelo de \textit{cap and trade} propuesto por \citeB{amigo_two_2021}. Como alcance final, se tiene, implementar el concepto de atención racional en el modelo como mínimo en una reformulación del modelo y proporcionar su resolución y compración con el modelo original. 
\vspace{2.5mm}

Debido a que este este es un concepto novedoso, queda fuera del alcance de este trabajo la proporción de un modelo con atención racional de implementación real en Chile. Se espera obtener resultados suficientes para incentivar su mejora y posterior versión con implicaciones reales.

\section{Metodología}
Con el fin de cumplir los objetivos y alcances de este trabajo se lleva a cabo la siguiente metodología:

\begin{enumerate}
\item Identificar modelos de equilibrio y modelos de equilibrio en capacidad: con bibliografía recomendad por el profesor guía y bibliografía metódicamente filtrada según estándares descritos según publicador y otras características, se lleva a cabo un estudio de los modelos de equilibrio.

\item Identificar métodos de resolución de MCP : para resolver los problemas de equilibrio es posible transformarlos en MCP al aplicar las condiciones de KKT. Se estudió principalmente de un curso realizado por Felipe Feijoo en el verano del año 2022 llamado "Computo de modelos de equilibrio", organizado por Sebastián Cea y la facultad de ingeniería de la Universidad de los Andes.   

\item Programar problemas MCP: es necesario aprender sobre distintos solvers de MCP y softwares que lo soporten. Primero, se realizan ensayos de programación con problemas de similar naturaleza al de \citeB{amigo_two_2021} en \citeB{d__aertrycke_risk_2017}. Finalmente, se busca profundizar con el curso de Felipe Feijoo sobre MCP. 

\item Replicar modelo original de Amigo: se desarrollan las condiciones de kkt y formulación de MCP para resolver el modelo original. Luego, se ejecuta el código original con los parámetros utilizados en el \textit{paper} para visualizar resultados y entender el funcionamiento.

\item Implementar el concepto de atención racional al modelo: se reformula el modelo incoporando el concepto de atención racional.

\item Evaluar nuevo modelo en \textit{solver} y analizar resultados: se evalúa el nuevo modelo al ejecutarlo en un \textit{solver} y se analizan las variables óptimas encontradas.
\end{enumerate}


\section{Estructura del documento}

La estructura del documento se divide en cinco capítulos: introducción, marco teórico, desarrollo metodológico, análisis de resultados y conclusiones.
\vspace{2.5mm}

La introducción tiene como objetivo contextualizar al lector sobre las implicaciones de este trabajo como también un resumen de todo lo realizado.
\vspace{2.5mm}

El marco teórico presenta todo el estudio realizado anterior a la implementación del concepto de atención racional en el modelo. En este se definen los problemas de equilibrio, MCP y la condiciones de KKT. Se presenta un tutorial para resolver problemas no lineales en distintos lenguajes de programación y \textit{solvers}. También se explica el trabajo realizado por \citeB{amigo_two_2021} y el concepto de atención racional.
\vspace{2.5mm}

En el desarrollo metodológico se replica el modelo original y luego se reformula en dos versiones nuevas. En adición a esto, se realiza la traducción del problema original al lenguaje de programación Julia con el fin de agilizar futuras resoluciones.
\vspace{2.5mm}

El capítulo 4 muestra los resultados de la replicación del modelo original y de los nuevos modelos formulados en el capítulo anterior. Se proporciona un análisis de estos y su comparación con el modelo original.
\vspace{2.5mm}

Finalmente, en el capítulo 5, se concluye si las nuevas versiones del modelo, que incorporan el concepto de atención racional, presenta resultados suficientes para su posterior mejora y presentación de un modelo de implementación real en Chile.




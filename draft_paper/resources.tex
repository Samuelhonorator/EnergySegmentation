%##################PACKAGE DECLARATIONS##################

\makeatletter
\newcommand*{\inlineequation}[2][]{%
  \begingroup
    % Put \refstepcounter at the beginning, because
    % package `hyperref' sets the anchor here.
    \refstepcounter{equation}%
    \ifx\\#1\\%
    \else
      \label{#1}%
    \fi
    % prevent line breaks inside equation
    \relpenalty=10000 %
    \binoppenalty=10000 %
    \ensuremath{%
      % \displaystyle % larger fractions, ...
      #2%
    }%
    ~\@eqnnum
  \endgroup
}
\makeatother

\let\proof\relax
\let\endproof\relax
\usepackage{amssymb,amsthm,color,enumerate,graphicx,mathrsfs,xspace,algorithm,algorithmicx,algpseudocode,booktabs,adjustbox,soul,threeparttable,pdflscape,afterpage,longtable,comment,multirow,etoolbox,subcaption}
\usepackage[usenames,dvipsnames]{xcolor}
\usepackage[noblocks,auth-sc]{authblk}
\usepackage[sort&compress,comma, numbers]{natbib}
\usepackage[colorlinks=true, allcolors=blue]{hyperref}
\usepackage[nameinlink]{cleveref}
%\usepackage[inline]{showlabels}
\usepackage[inline]{enumitem}
%##################PACKAGE DECLARATIONS##################
\newcommand{\repthanks}[1]{\textsuperscript{\ref{#1}}}
\renewcommand{\qedsymbol}{$\blacksquare$}
\renewcommand{\bar}{\overline}
\renewcommand{\tilde}{\widetilde}
\renewcommand{\hat}{\widehat}
\renewcommand{\algorithmicrequire}{\textbf{Input:}}
\renewcommand{\algorithmicensure}{\textbf{Output:}}
%COLORS
\colorlet{rosso}{red!90!white}
\definecolor{green}{RGB}{45, 204, 53}
\definecolor{blue}{RGB}{59, 42, 247}
\newcommand{\sri}[1]{\textbf{\textcolor{rosso}{[Sriram: #1]}}}
\newcommand{\felipe}[1]{\textbf{\textcolor{green}{[Felipe: #1]}}}
\newcommand{\gabri}[1]{\textbf{\textcolor{blue}{[Gabriele: #1]}}}
\newcommand{\mc}[1]{\textbf{\textcolor{teal}{[Margarida: #1]}}}
\newcommand{\andrea}[1]{\textbf{\textcolor{pink}{[Andrea #1]}}}
\newtheorem{Ex}{Example}
\newtheorem{Def}{Definition}
\newtheorem{Cla}{Claim}
\newtheorem{Corr}[Cla]{Corollary}
\allowdisplaybreaks
\crefname{lemma}{Lemma}{Lemmata}
\crefname{theorem}{Theorem}{Theorems}
\crefname{claim}{Claim}{Claims}
\crefname{algorithm}{Algorithm}{Algorithms}
\crefname{equation}{}{}
\crefname{Def}{Definition}{Definition}
\crefname{Cla}{Claim}{Claim}
\crefname{Corr}{Corollary}{Corollaries}
\crefname{remark}{Remark}{Remarks}
\crefname{Ex}{Example}{Examples}
\crefname{figure}{Figure}{Figures}
\crefname{section}{Section}{Sections}
\crefname{table}{Table}{Tables}
\crefname{enumi}{Statement}{Statements}
\crefname{line}{Step}{Steps}

\newcommand{\SSI}{\texttt{SUBSET SUM INTERVAL}\xspace}
\newcommand{\PNE}{\emph{PNE}\xspace}
\newcommand{\PNEs}{\emph{PNE}s\xspace}
\newcommand{\NE}{\emph{NE}\xspace}
\newcommand{\NEs}{\emph{NE}s\xspace}
\newcommand{\MNE}{\emph{MNE}\xspace}
\newcommand{\MNEs}{\emph{MNE}s\xspace}
\newcommand{\NASP}{\emph{NASP}\xspace}
\newcommand{\NASPs}{\emph{NASP}s\xspace}
\newcommand{\LCP}{\emph{LCP}\xspace}
\newcommand{\LCPs}{\emph{LCP}s\xspace}
\newcommand{\SPr}[1][x]{SPr($#1$)\xspace}
\newcommand{\NPC}{$\mathcal{NP}$-complete\xspace}

\newcommand{\q}		{\mathbf{\textcolor{blue}{q}}}
\newcommand{\Cost}	{\mathbf{\textcolor{rosso}{C}}}
\newcommand{\CostQ}	{\mathbf{\textcolor{rosso}{D}}}
\newcommand{\ith} {$i^{th}$}
\newcommand{\taxb}  {\textcolor{rosso}{b}}
\newcommand{\Cemm}[1][p] {\Cost^{#1}_{\text{emmision}}}
\newcommand{\qi}[1][p] 	{\q^{#1}}
\newcommand{\qiCap}[1][p] 	{\overline{\mathbf{\textcolor{rosso}{q^{#1}}}}}
\newcommand{\ti}[1][p] 	{\mathbf{\textcolor{blue}{t}}^{#1}}
\newcommand{\tiCap}[1][p] 	{\overline{\mathbf{\textcolor{rosso}{t}}^{#1}}}
\newcommand{\piFloor}[1][C] 	{\underline{\mathbf{\textcolor{rosso}{\pi}}^{#1}}}
\newcommand{\DemInt}[1][C] {\textcolor{rosso}{\alpha^{#1}}}
\newcommand{\DemSlope}[1][C] {\textcolor{rosso}{\beta^{#1}}}
\newcommand{\qimp}[1][C] {\q^{#1}_{\text{imp}}}
\newcommand{\qimpI}[1][I\to A] {\q^{#1}_{\text{imp}}}
\newcommand{\qexp}[1][C] {\q^{#1}_{\text{exp}}}
\newcommand{\Cilin}[1][p]	{\Cost_{#1}}
\newcommand{\Ciquad}[1][p]	{\CostQ_{#1}}
\newcommand{\qAimp} {\q^A_{\text{imp}}}
\newcommand{\qAimpI}[1][I\to A] {\q^{#1}_{\text{imp}}}
\newcommand{\qAexp} {\q^A_{\text{exp}}}
\newcommand{\qAi}[1][i] 	{\q^A_{#1}}
\newcommand{\piI}[1][I] {\mathbf{\textcolor{green}{\pi}}^{#1}}
\newcommand{\tAi} 	{\mathbf{\textcolor{blue}{t}}^A_i}
\newcommand{\CAilin}	{\Cost^{A}_i}
\newcommand{\CAiquad}	{\CostQ^{A}_i}
\newcommand{\R}{\mathbb{R}}
\newcommand{\Q}{\mathbb{Q}}
\newcommand{\Z}{\mathbb{Z}}
\newcommand{\vecN}[1]{ \left(\begin{array}{c}#1\end{array}\right) }
\newcommand{\vecNT}[1]{ \left(\begin{array}#1\end{array}\right) }

% Operators that might be of use - Uncomment as needed
\newcommand{\sol}{\operatorname{SOL}}
\newcommand{\conv}{\operatorname{conv}}
\newcommand{\intr}{\operatorname{int}}
\newcommand{\relintr}{\operatorname{relint}}
\newcommand{\relint}{\relintr}
\newcommand{\rec}{\operatorname{rec}}
\newcommand{\lin}{\operatorname{lin}}
\newcommand{\bd}{\operatorname{bd}}
\newcommand{\cl}{\operatorname{cl}}
\newcommand{\proj}{\operatorname{proj}}
\newcommand{\sgn}{\operatorname{sgn}}
\newcommand{\cone}{\operatorname{cone}}

\newcommand{\floor}[1]{\left\lfloor#1\right\rfloor}
\newcommand{\ceil}[1]{\left\lceil#1\right\rceil}
\definecolor{mycolor}{RGB}{200,0,0}
%\renewcommand{\showlabelfont}{\tiny\bf\color{mycolor}}

\newtoggle{ExtendedVersion}  
%##################COMMAND DECLARATIONS##################


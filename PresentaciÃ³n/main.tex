\documentclass{beamer}
%
% Choose how your presentation looks.
%
% For more themes, color themes and font themes, see:
% http://deic.uab.es/~iblanes/beamer_gallery/index_by_theme.html
%
\mode<presentation>
{
  \usetheme{default}      % or try Darmstadt, Madrid, Warsaw, ...
  \usecolortheme{default} % or try albatross, beaver, crane, ...
  \usefonttheme{default}  % or try serif, structurebold, ...
  \setbeamertemplate{navigation symbols}{}
  \setbeamertemplate{caption}[numbered]
} 

\usepackage[english]{babel}
\usepackage[utf8x]{inputenc}

\title[Your Short Title]{Electric Markets and Restricted Participation}
\author{Amigo, Cea, Feijoo}
\institute{EII - PUCV}
\date{2019}

\begin{document}

\begin{frame}
  \titlepage
\end{frame}

% Uncomment these lines for an automatically generated outline.
%\begin{frame}{Outline}
%  \tableofcontents
%\end{frame}

\section{Notation}

\begin{frame}{Notation}


\begin{itemize}
    \item $x \in \mathbb{R}^n$
    \item Uncertain outcomes (costs): $\mathcal{Z} := \mathbb{R}^K$
    \begin{itemize}
        \item[-]stochastic scenarios $\omega \in \Omega:= \{ 1, ..., K \}$
        \item[-]member $Z$ of  $\mathcal{Z}$ is $Z_{\omega}$
    \end{itemize}
    \item Probability density $\Pi \in \mathbb{R}^n$
    \begin{itemize}
        \item[-] set of probability measures is $\mathcal{P}$
    \end{itemize}
\end{itemize}



\vskip 0.5cm

\begin{block}{Risk Neutral case}
Evaluate the cost of any $Z \in \mathcal{Z}$ as
$\mathbb{E}[Z] := \Pi [Z] := \sum\limits_{\omega} \Pi_{\omega} Z_{\omega}$

{\footnotesize In real cases, data is associated with a real world probability measure $\Theta$}
\end{block}

\begin{block}{Coherent risk measures}
Each $r_i : \mathcal{Z} \rightarrow \mathbb{R}$ is a CRM with nonempty closed convex risk set $\mathcal{D}_i$ (agent i's risk set), i.e, $r_i(Z) = \max_{\Pi \in \mathcal{D}_i} \mathbb{E}_\Pi [Z]$ for each $Z \in \mathcal{Z}$. 


\end{block}

\end{frame}

\section{Two stage stochastic competitive capacity equilibrium problems}

\begin{frame}{Two stage stochastic competitive capacity equilibrium problems}
Noncooperative Nash game of $N$ agents
\begin{itemize}
    \item strategy set for agent $i$ $X_i \subset \mathbb{R}^{n_i}$. $n_i$: dimension of investment variables of agent $i$
    \item vector of design variables $x_i$: agent $i$'s investment (capacity)  in a risky asset $\Xi_i$ (cost from production plant to operate in an uncertain market): $\Xi_i(x_i,x_{-i}):= (\Xi_{i\omega}(x_i, x_{-i})) \in \mathcal{Z}$ \textbf{(First stage)}
    \item Each scenario cost $\Xi_{i\omega}(x_i, x_{-i})$ is the optimal value of a production optimization problem carried out by agent $i$
    \item agent $i$ chooses its \textbf{second stage } production quantity $y_i$ given the plant capacity and operating costs parameters determined in first stage, i.e., $x_i$ and $x_{-i}$
\end{itemize}
    
\end{frame}

\begin{frame}{Two stage stochastic competitive capacity equilibrium problems}
    \begin{itemize}
        \item basket of financial products: vector $W_i \in \mathcal{W}$, where $\mathcal{W} \subset
        \mathcal{Z}$
        \item hedges the risk in the form $r_i(\Xi_i(x_i,x_{-i})-W_i)$
        
    \end{itemize}
    
    
    \begin{block}{Agent $i$'s optimization}
    
    $$\min_{x_i,W_i}P^r$$
    
    \end{block}
\end{frame}

%\section{Some \LaTeX{} Examples}

%\subsection{Tables and Figures}

%\begin{frame}{Tables and Figures}

%\begin{itemize}
%\item Use \texttt{tabular} for basic tables --- see Table~\ref{tab:widgets}, for example.
%\item You can upload a figure (JPEG, PNG or PDF) using the files menu. 
%\item To include it in your document, use the \texttt{includegraphics} command (see the comment below in the source code).
%\end{itemize}

% Commands to include a figure:
%\begin{figure}
%\includegraphics[width=\textwidth]{your-figure's-file-name}
%\caption{\label{fig:your-figure}Caption goes here.}
%\end{figure}

%\begin{table}
%\centering
%\begin{tabular}{l|r}
%Item & Quantity \\\hline
%Widgets & 42 \\
%Gadgets & 13
%\end{tabular}
%\caption{\label{tab:widgets}An example table.}
%\end{table}

%\end{frame}

%\subsection{Mathematics}

%\begin{frame}{Readable Mathematics}

%Let $X_1, X_2, \ldots, X_n$ be a sequence of independent and identically distributed random variables with $\text{E}[X_i] = \mu$ and $\text{Var}[X_i] = \sigma^2 < \infty$, and let
%$$S_n = \frac{X_1 + X_2 + \cdots + X_n}{n}
%      = \frac{1}{n}\sum_{i}^{n} X_i$$
%denote their mean. Then as $n$ approaches infinity, the random variables $\sqrt{n}(S_n - \mu)$ converge in distribution to a normal $\mathcal{N}(0, \sigma^2)$.

%\end{frame}

\end{document}
